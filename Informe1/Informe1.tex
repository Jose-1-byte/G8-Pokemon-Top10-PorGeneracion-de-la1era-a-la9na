% Options for packages loaded elsewhere
\PassOptionsToPackage{unicode}{hyperref}
\PassOptionsToPackage{hyphens}{url}
\documentclass[
]{article}
\usepackage{xcolor}
\usepackage[margin=1in]{geometry}
\usepackage{amsmath,amssymb}
\setcounter{secnumdepth}{-\maxdimen} % remove section numbering
\usepackage{iftex}
\ifPDFTeX
  \usepackage[T1]{fontenc}
  \usepackage[utf8]{inputenc}
  \usepackage{textcomp} % provide euro and other symbols
\else % if luatex or xetex
  \usepackage{unicode-math} % this also loads fontspec
  \defaultfontfeatures{Scale=MatchLowercase}
  \defaultfontfeatures[\rmfamily]{Ligatures=TeX,Scale=1}
\fi
\usepackage{lmodern}
\ifPDFTeX\else
  % xetex/luatex font selection
\fi
% Use upquote if available, for straight quotes in verbatim environments
\IfFileExists{upquote.sty}{\usepackage{upquote}}{}
\IfFileExists{microtype.sty}{% use microtype if available
  \usepackage[]{microtype}
  \UseMicrotypeSet[protrusion]{basicmath} % disable protrusion for tt fonts
}{}
\makeatletter
\@ifundefined{KOMAClassName}{% if non-KOMA class
  \IfFileExists{parskip.sty}{%
    \usepackage{parskip}
  }{% else
    \setlength{\parindent}{0pt}
    \setlength{\parskip}{6pt plus 2pt minus 1pt}}
}{% if KOMA class
  \KOMAoptions{parskip=half}}
\makeatother
\usepackage{graphicx}
\makeatletter
\newsavebox\pandoc@box
\newcommand*\pandocbounded[1]{% scales image to fit in text height/width
  \sbox\pandoc@box{#1}%
  \Gscale@div\@tempa{\textheight}{\dimexpr\ht\pandoc@box+\dp\pandoc@box\relax}%
  \Gscale@div\@tempb{\linewidth}{\wd\pandoc@box}%
  \ifdim\@tempb\p@<\@tempa\p@\let\@tempa\@tempb\fi% select the smaller of both
  \ifdim\@tempa\p@<\p@\scalebox{\@tempa}{\usebox\pandoc@box}%
  \else\usebox{\pandoc@box}%
  \fi%
}
% Set default figure placement to htbp
\def\fps@figure{htbp}
\makeatother
\setlength{\emergencystretch}{3em} % prevent overfull lines
\providecommand{\tightlist}{%
  \setlength{\itemsep}{0pt}\setlength{\parskip}{0pt}}
\usepackage{bookmark}
\IfFileExists{xurl.sty}{\usepackage{xurl}}{} % add URL line breaks if available
\urlstyle{same}
\hypersetup{
  pdftitle={Análisis Estadístico y Ranking de los Pokémon Más Destacados por Generación},
  pdfauthor={Sofia Rodriguez,Jose Aguirre y Carmen De Bolivar},
  hidelinks,
  pdfcreator={LaTeX via pandoc}}

\title{Análisis Estadístico y Ranking de los Pokémon Más Destacados por
Generación}
\author{Sofia Rodriguez,Jose Aguirre y Carmen De Bolivar}
\date{2025-02-27}

\begin{document}
\maketitle

\subsection{Planteamiento del
Problema}\label{planteamiento-del-problema}

Desde su creación, la franquicia Pokémon ha presentado una gran
diversidad de especies, cada una con atributos y características únicas.
Con más de 1000 Pokémon a la fecha, resulta complejo determinar
objetivamente cuáles sobresalen en cada generación. Los rankings
existentes a menudo incorporan sesgos subjetivos, como la popularidad o
el diseño de los Pokémon. Esta investigación busca superar estas
limitaciones mediante el uso de análisis estadísticos descriptivos para
identificar y clasificar a los Pokémon más destacados de cada
generación, basándose en sus pricipales estadísticas base.

\subsection{Justificación de la
Investigación}\label{justificaciuxf3n-de-la-investigaciuxf3n}

Este proyecto proporcionará a la comunidad de jugadores y entusiastas
una evaluación objetiva y basada en datos sobre los Pokémon más
sobresalientes de cada generación. El análisis estadístico permitirá
identificar patrones y tendencias de los mejores Pokémon en cada
generación, así como comprender mejor algunas las mecánicas internas del
juego, los resultados podrán ser utilizados para optimizar equipos y
estrategias de juego. Además, este proyecto servirá como una herramienta
educativa para demostrar la aplicación de técnicas de las estadísticas
descriptivas en un contexto más amigable y divertido para la persona
promedio.

\subsection{Límites y Alcances}\label{luxedmites-y-alcances}

El análisis se limitará a las estadísticas base de los Pokémon,
incluyendo HP (puntos de salud), ataque, defensa, ataque especial,
defensa especial y velocidad. Se utilizarán técnicas de estadística
descriptiva para analizar y comparar los datos. El estudio abarcará
todas las nueve generaciones de Pokémon presentes en la base de datos
proporcionada. No se considerarán factores externos como habilidades,
movimientos, estrategias de combate o tiers competitivos.

\subsection{Marco Teórico}\label{marco-teuxf3rico}

El proyecto se fundamentará en los conceptos de estadística descriptiva,
incluyendo medidas de tendencia central, dispersión y tipificando los
datos. Se utilizarán herramientas de programación como R para el
análisis y visualización de los datos. Al final de cada top se tomarán
en cuenta los tipos de Pokémon y sus habilidades más importantes a la
hora de competir contra un líder de gimnasio.

\subsection{Explicación de las
Variables:}\label{explicaciuxf3n-de-las-variables}

\begin{itemize}
\tightlist
\item
  id: Identificador único del Pokémon.
\item
  name: Nombre del Pokémon.
\item
  generation: Generación a la que pertenece el Pokémon.
\item
  evolves\_from: Pokémon del que evoluciona, si aplica.
\item
  type1: Tipo primario del Pokémon.
\item
  type2: Tipo secundario del Pokémon, si aplica.
\item
  hp: Puntos de salud base del Pokémon.
\item
  atk: Ataque base del Pokémon.
\item
  def: Defensa base del Pokémon.
\item
  spatk: Ataque especial base del Pokémon.
\item
  spdef: Defensa especial base del Pokémon.
\item
  speed: Velocidad base del Pokémon.
\item
  total: Suma de todas las estadísticas base del Pokémon (HP, ataque,
  defensa, ataque especial, defensa especial y velocidad). Esta variable
  será fundamental para el ranking.
\item
  height: Altura del Pokémon.
\item
  weight: Peso del Pokémon.
\item
  abilities: Habilidades del Pokémon.
\item
  desc: Descripción del Pokémon.
\end{itemize}

\subsection{Objetivos}\label{objetivos}

\subsubsection{General}\label{general}

Desarrollar un informe dinámico en RMarkdown que, mediante análisis
estadísticos, determine el top 5 de los Pokémon más destacados por
generación, utilizando una base de datos con información de las nueve
generaciones y sus principales estadísticas.

\subsubsection{Específicos:}\label{especuxedficos}

\begin{itemize}
\tightlist
\item
  Dividir la base de datos en subconjuntos en base a cada generación de
  Pokémon.
\item
  Realizar un análisis descriptivo de las estadísticas de manera general
  para todos los pokemones.
\item
  Realizar la tipificación de los datos de las estadísticas de manera
  general general para todos los pokemones.
\item
  Ordenar los Pokémon de cada generación tomando en cuenta los datos
  tipificados de mayor a menor.
\item
  Seleccionar los 5 mejores Pokémon de cada generación, basándonos en
  los datos tipificados, para formar el ranking.
\item
  Mencionar la efectividad de los pokemons contra los lideres de
  gimnacio (opcional)
\item
  Visualizar los resultados en tablas y gráficos generados en RMarkdown,
  que permitan la comparación entre generaciones y la identificación de
  los Pokémon más destacados.
\item
  Desarrollar un dashboard interactivo que resuma gráficamente la
  información más relevante del estudio.
\end{itemize}

\section{Contexto General del Tema}\label{contexto-general-del-tema}

Desde el lanzamiento de la primera generación de Pokémon en 1996,
evaluar objetivamente el desempeño de estas criaturas ha sido un desafío
constante. Inicialmente, los jugadores comparaban las especies
utilizando criterios cualitativos como el diseño o la rareza, o mediante
métricas no estandarizadas como el Total de Estadísticas Base (BST, por
sus siglas en inglés), que es la suma de las seis estadísticas
fundamentales: Puntos de Salud (HP), Ataque, Defensa, Ataque Especial,
Defensa Especial y Velocidad. Sin embargo, este enfoque presenta dos
problemas críticos:

\begin{itemize}
\tightlist
\item
  \textbf{Escalada de Poder (Power Creep)}: Las generaciones más
  recientes introducen Pokémon con BST significativamente más altos. Por
  ejemplo, el BST promedio de la Generación I es 420, mientras que en la
  Generación IX supera los 480. Esta diferencia crea una brecha
  artificial entre especies antiguas y nuevas, dificultando las
  comparaciones directas sin ajustar por variaciones generacionales
  (Bulbapedia, 2023).
\item
  \textbf{Heterogeneidad de Tipos}: La distribución de tipos de Pokémon
  no es uniforme a lo largo de las generaciones. Algunos tipos, como
  Agua o Volador, son más comunes, mientras que otros, como Dragón, son
  menos frecuentes. Esta disparidad puede distorsionar los rankings no
  ajustados, ya que la abundancia o escasez de ciertos tipos influye en
  el desempeño relativo y en las estrategias disponibles para los
  jugadores (Pokémon Database, 2023).
\end{itemize}

Este estudio se enmarca en el ámbito de la estadística descriptiva
estandarizada, un enfoque que permite neutralizar sesgos sistémicos al
comparar Pokémon de diferentes generaciones. La relevancia de este
enfoque radica en ofrecer un método replicable y transparente para
establecer rankings intergeneracionales basados exclusivamente en
métricas objetivas.

\section{Revisión de Investigaciones
Previas}\label{revisiuxf3n-de-investigaciones-previas}

\subsection{a) Métodos Cuantitativos Sin
Estandarización}\label{a-muxe9todos-cuantitativos-sin-estandarizaciuxf3n}

\begin{itemize}
\tightlist
\item
  \textbf{Listas Basadas en BST Crudo}: Sitios especializados como
  Serebii.net (2001 - actualidad) han publicado rankings según el BST
  sin considerar ajustes estadísticos. Este enfoque destaca Pokémon como
  Slaking (Generación III, BST 670), sin tener en cuenta limitaciones
  específicas que afectan su rendimiento real en combate, como su
  habilidad Truant (Serebii.net, 2023).
\item
  \textbf{Documentación de Estadísticas por Generación}: Bulbapedia
  (2005 - actualidad) ha recopilado y presentado las estadísticas base
  de los Pokémon por generación. Sin embargo, estas compilaciones no
  realizan ajustes por diferencias en las medias o desviaciones estándar
  entre generaciones, lo que dificulta comparaciones precisas y justas
  entre Pokémon de distintas épocas (Bulbapedia, 2023).
\end{itemize}

\subsection{Brechas Identificadas}\label{brechas-identificadas}

\begin{itemize}
\tightlist
\item
  \textbf{Falta de Normalización Estadística}: Comparar un BST de 500 en
  la Generación I (media 420) con uno en la Generación IX (media 480) es
  engañoso sin realizar ajustes estadísticos. Sin normalización, las
  diferencias naturales en las estadísticas base promedio de cada
  generación pueden llevar a conclusiones inexactas sobre la verdadera
  fortaleza de un Pokémon en su contexto generacional.
\item
  \textbf{Variables No Consideradas}: La frecuencia y distribución de
  tipos entre generaciones afectan la utilidad y el desempeño real de
  los Pokémon. Rankings que no cuantifican y ajustan estos factores
  pueden ofrecer resultados sesgados, omitiendo cómo la abundancia o
  escasez de ciertos tipos influye en el metajuego y en las estrategias
  disponibles (Pokémon Database, 2023).
\end{itemize}

\section{Aporte de Esta
Investigación}\label{aporte-de-esta-investigaciuxf3n}

Este estudio busca superar las limitaciones anteriores mediante:

\begin{itemize}
\item
  \textbf{Aplicación de Puntuaciones Z (Z-scores)}: Utilizaremos la
  fórmula estándar de puntuación Z:

  \[z = \frac{X - \mu}{\sigma}\] Donde:
\item
  \(X\) es el valor individual.
\item
  \(\mu\) es la media.
\item
  \(\sigma\) es la desviación estándar.
\item
  \textbf{Enfoque en Métricas Objetivas y Replicables}: Al centrarnos
  exclusivamente en las estadísticas base estandarizadas y evitar
  factores subjetivos como habilidades, movimientos o popularidad,
  garantizamos que el ranking resultante sea imparcial y basado en datos
  cuantitativos sólidos.
\end{itemize}

\section{Importancia y Contribución del
Estudio}\label{importancia-y-contribuciuxf3n-del-estudio}

Al ofrecer un ranking objetivo de los mejores Pokémon por generación
basado en análisis estadísticos estandarizados, este estudio:

\begin{itemize}
\tightlist
\item
  \textbf{Proporciona una Evaluación Imparcial}: Identifica de manera
  precisa cuáles Pokémon sobresalen dentro de su generación y en
  comparación con otras, basándose en su desempeño estadístico relativo.
\item
  \textbf{Facilita Comparaciones Intergeneracionales}: La normalización
  estadística elimina los sesgos introducidos por la escalada de poder,
  permitiendo comparaciones equitativas entre Pokémon de distintas
  generaciones.
\item
  \textbf{Apoya a la Comunidad de Jugadores y Analistas}: Ofrece
  información valiosa para jugadores que buscan optimizar sus equipos y
  estrategias, así como para analistas interesados en la evolución del
  diseño y balance en la franquicia Pokémon.
\item
  \textbf{Contribuye al Campo de Estudios de Videojuegos}: Demuestra la
  aplicación práctica de métodos estadísticos en el análisis de
  videojuegos, fomentando investigaciones futuras que apliquen enfoques
  similares en otros contextos.
\end{itemize}

\section{Referencias}\label{referencias}

\begin{itemize}
\tightlist
\item
  Bulbapedia. (2023). Base Stat Total (BST). Recuperado el 15 de octubre
  de 2023, de
  \href{https://bulbapedia.bulbagarden.net/}{https://bulbapedia.bulbagarden.net}
\item
  Pokémon Database. (2023). Type distribution by generation. Recuperado
  el 15 de octubre de 2023, de \url{https://pokemondb.net/type}
\item
  Serebii.net. (2023). Base stat averages by generation. Recuperado el
  15 de octubre de 2023, de
  \href{https://www.serebii.net/}{https://www.serebii.net}
\item
  Smogon University. (2023). Smogon tier lists. Recuperado el 15 de
  octubre de 2023, de
  \href{https://www.smogon.com/}{https://www.smogon.com}
\item
  The Pokémon Company. (2020). Pokémon of the Year results. Recuperado
  el 15 de octubre de 2023, de
  \url{https://www.pokemon.com/es/noticias/pokemon-of-the-year}
\end{itemize}

\section{Abordamiento}\label{abordamiento}

Este trabajo aborda la necesidad de una evaluación objetiva y justa de
los Pokémon a través de las diferentes generaciones. Al aplicar métodos
estadísticos estandarizados y centrarse en métricas cuantitativas
replicables, proporcionamos un ranking que refleja con mayor precisión
el desempeño relativo de las especies. Este enfoque no solo beneficia a
los jugadores y entusiastas de la franquicia, sino que también
contribuye al ámbito académico al demostrar cómo la estadística
descriptiva puede aplicarse eficazmente en el análisis de videojuegos.

\end{document}
