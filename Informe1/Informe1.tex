% Options for packages loaded elsewhere
\PassOptionsToPackage{unicode}{hyperref}
\PassOptionsToPackage{hyphens}{url}
\documentclass[
]{article}
\usepackage{xcolor}
\usepackage[margin=1in]{geometry}
\usepackage{amsmath,amssymb}
\setcounter{secnumdepth}{-\maxdimen} % remove section numbering
\usepackage{iftex}
\ifPDFTeX
  \usepackage[T1]{fontenc}
  \usepackage[utf8]{inputenc}
  \usepackage{textcomp} % provide euro and other symbols
\else % if luatex or xetex
  \usepackage{unicode-math} % this also loads fontspec
  \defaultfontfeatures{Scale=MatchLowercase}
  \defaultfontfeatures[\rmfamily]{Ligatures=TeX,Scale=1}
\fi
\usepackage{lmodern}
\ifPDFTeX\else
  % xetex/luatex font selection
\fi
% Use upquote if available, for straight quotes in verbatim environments
\IfFileExists{upquote.sty}{\usepackage{upquote}}{}
\IfFileExists{microtype.sty}{% use microtype if available
  \usepackage[]{microtype}
  \UseMicrotypeSet[protrusion]{basicmath} % disable protrusion for tt fonts
}{}
\makeatletter
\@ifundefined{KOMAClassName}{% if non-KOMA class
  \IfFileExists{parskip.sty}{%
    \usepackage{parskip}
  }{% else
    \setlength{\parindent}{0pt}
    \setlength{\parskip}{6pt plus 2pt minus 1pt}}
}{% if KOMA class
  \KOMAoptions{parskip=half}}
\makeatother
\usepackage{graphicx}
\makeatletter
\newsavebox\pandoc@box
\newcommand*\pandocbounded[1]{% scales image to fit in text height/width
  \sbox\pandoc@box{#1}%
  \Gscale@div\@tempa{\textheight}{\dimexpr\ht\pandoc@box+\dp\pandoc@box\relax}%
  \Gscale@div\@tempb{\linewidth}{\wd\pandoc@box}%
  \ifdim\@tempb\p@<\@tempa\p@\let\@tempa\@tempb\fi% select the smaller of both
  \ifdim\@tempa\p@<\p@\scalebox{\@tempa}{\usebox\pandoc@box}%
  \else\usebox{\pandoc@box}%
  \fi%
}
% Set default figure placement to htbp
\def\fps@figure{htbp}
\makeatother
\setlength{\emergencystretch}{3em} % prevent overfull lines
\providecommand{\tightlist}{%
  \setlength{\itemsep}{0pt}\setlength{\parskip}{0pt}}
\usepackage{bookmark}
\IfFileExists{xurl.sty}{\usepackage{xurl}}{} % add URL line breaks if available
\urlstyle{same}
\hypersetup{
  pdftitle={Informe1},
  pdfauthor={Sofia},
  hidelinks,
  pdfcreator={LaTeX via pandoc}}

\title{Informe1}
\author{Sofia}
\date{2025-02-23}

\begin{document}
\maketitle

\subsection{Planteamiento del
Problema}\label{planteamiento-del-problema}

Desde su creación, la franquicia Pokémon ha presentado una gran
diversidad de especies, cada una con atributos y características únicas.
Con más de 1000 Pokémon a la fecha, resulta complejo determinar
objetivamente cuáles sobresalen en cada generación. Los rankings
existentes a menudo incorporan sesgos subjetivos, como la popularidad o
el diseño de los Pokémon. Esta investigación busca superar estas
limitaciones mediante el uso de análisis estadísticos descriptivos para
identificar y clasificar a los Pokémon más destacados de cada
generación, basándose en sus pricipales estadísticas base.

\subsection{Justificación de la
Investigación}\label{justificaciuxf3n-de-la-investigaciuxf3n}

Este proyecto proporcionará a la comunidad de jugadores y entusiastas
una evaluación objetiva y basada en datos sobre los Pokémon más
sobresalientes de cada generación. El análisis estadístico permitirá
identificar patrones y tendencias de los mejores Pokémon en cada
generación, así como comprender mejor algunas las mecánicas internas del
juego, los resultados podrán ser utilizados para optimizar equipos y
estrategias de juego. Además, este proyecto servirá como una herramienta
educativa para demostrar la aplicación de técnicas de las estadísticas
descriptivas en un contexto más amigable y divertido para la persona
promedio.

\subsection{Límites y Alcances}\label{luxedmites-y-alcances}

El análisis se limitará a las estadísticas base de los Pokémon,
incluyendo HP (puntos de salud), ataque, defensa, ataque especial,
defensa especial y velocidad. Se utilizarán técnicas de estadística
descriptiva para analizar y comparar los datos. El estudio abarcará
todas las nueve generaciones de Pokémon presentes en la base de datos
proporcionada. No se considerarán factores externos como habilidades,
movimientos, estrategias de combate o tiers competitivos.

\subsection{Marco Teórico}\label{marco-teuxf3rico}

El proyecto se fundamentará en los conceptos de estadística descriptiva,
incluyendo medidas de tendencia central, dispersión y tipificando los
datos. Se utilizarán herramientas de programación como R para el
análisis y visualización de los datos. Al final de cada top se tomarán
en cuenta los tipos de Pokémon y sus habilidades más importantes a la
hora de competir contra un líder de gimnasio.

\subsection{Definición de las
Variables:}\label{definiciuxf3n-de-las-variables}

\begin{itemize}
\tightlist
\item
  id: Identificador único del Pokémon.
\item
  name: Nombre del Pokémon.
\item
  generation: Generación a la que pertenece el Pokémon.
\item
  evolves\_from: Pokémon del que evoluciona, si aplica.
\item
  type1: Tipo primario del Pokémon.
\item
  type2: Tipo secundario del Pokémon, si aplica.
\item
  hp: Puntos de salud base del Pokémon.
\item
  atk: Ataque base del Pokémon.
\item
  def: Defensa base del Pokémon.
\item
  spatk: Ataque especial base del Pokémon.
\item
  spdef: Defensa especial base del Pokémon.
\item
  speed: Velocidad base del Pokémon.
\item
  total: Suma de todas las estadísticas base del Pokémon (HP, ataque,
  defensa, ataque especial, defensa especial y velocidad). Esta variable
  será fundamental para el ranking.
\item
  height: Altura del Pokémon.
\item
  weight: Peso del Pokémon.
\item
  abilities: Habilidades del Pokémon.
\item
  desc: Descripción del Pokémon.
\end{itemize}

\subsection{Objetivos}\label{objetivos}

\subsubsection{General}\label{general}

Desarrollar un informe dinámico en RMarkdown que, mediante análisis
estadísticos, determine el top 5 de los Pokémon más destacados por
generación, utilizando una base de datos con información de las nueve
generaciones y sus principales estadísticas.

\subsubsection{Específicos}\label{especuxedficos}

\begin{itemize}
\tightlist
\item
  Dividir la base de datos en subconjuntos en base a cada generación de
  Pokémon.
\item
  Realizar un análisis descriptivo de las estadísticas de manera general
  para todos los pokemones.
\item
  Realizar la tipificación de los datos de las estadísticas de manera
  general general para todos los pokemones.
\item
  Ordenar los Pokémon de cada generación tomando en cuenta los datos
  tipificados de mayor a menor.
\item
  Seleccionar los 5 mejores Pokémon de cada generación, basándonos en
  los datos tipificados, para formar el ranking.
\item
  Mencionar la efectividad de los pokemons contra los lideres de
  gimnacio (opcional)
\item
  Visualizar los resultados en tablas y gráficos generados en RMarkdown,
  que permitan la comparación entre generaciones y la identificación de
  los Pokémon más destacados.
\item
  Desarrollar un dashboard interactivo que resuma gráficamente la
  información más relevante del estudio.
\end{itemize}

\end{document}
